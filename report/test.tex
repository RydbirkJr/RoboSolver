\section{Introduction}\label{introduction}

\subsection{Ricochet Robots}\label{ricochet-robots}

Beskrivelse af spillet - Bevæger sig som rooks i skak\\
- Din mor

Test - test

\subsection{Goals}\label{goals}

\subsection{Previous Solutions}\label{previous-solutions}

\subsection{Baseline}\label{baseline}

Et navn for mål-robotten Et navn for ikke-mål-robotterne

\section{Solutions}\label{solutions}

\subsection{JPS+}\label{jps}

\subsubsection{Algorithm}\label{algorithm}

\subsubsection{Analysis}\label{analysis}

\subsection{Graph-DP}\label{graph-dp}

\subsubsection{Algorithm}\label{algorithm-1}

The base idea of the graph based algorithm is to construct a graph for
the entire board with directional edges for each vertex and represent
this in a two dimensional array. Constructing the graph is done in two
passes with dynamic programming.

PSEUDO-kode?

A basic assumption in this solution is that all moves by the Obstacle
Robots, OR, can be applied before the actual Goal Robot, GR, moves
towards goal. Thus, the GR can be dependent on OR while OR can depend
all other OR states but only on GR's starting position.

Afterwards, the solution will move the OR in all possible directions,
evaluating the search tree with a branching factor of \(9^n\) where n is
the

A heuristic function is used to evaluate the current best solution
compared to the

\subsubsection{Analysis}\label{analysis-1}

\section{Experimental results}\label{experimental-results}

\subsection{Setup}\label{setup}

\subsection{Results}\label{results}
